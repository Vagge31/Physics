\documentclass[12pt]{article}
\usepackage{fontspec}
\setmainfont{CMU Serif}
\usepackage{amsmath,amsthm,amssymb, multicol, array}
\usepackage{polyglossia}
%\setdefaultlanguage{greek}
\setotherlanguages{english}
\usepackage{enumerate}
\usepackage{enumitem}
\usepackage{moreenum}
\usepackage[margin=1.5cm]{geometry}
\usepackage{graphicx}
%opening
\title{Υπολογιστική Φυσική \\ 1\textsuperscript{η} Εργασία}
\date{ }
\author{Πέππας Ευάγγελος (202200124)}

\begin{document}
\maketitle
\section*{Πρόβλημα 9 (Λύση)}
Το πρώτο βήμα για να λύσουμε το πρόβλημα είναι να μετασχηματίσουμε μέσω ενός μετασχηματισμού $T(U)$, τους τυχαίους αριθμούς που ακολουθούν την ομοιόμορφη κατανομή σε τυχαίους αριθμούς που ακολουθούν την κατανομή πυκνότητας πιθανότητας Rayleigh: \\
\[  f(x) = 
\begin{cases}
	\frac{x}{\sigma ^2} e^{-x^2/2\sigma^2} & x \geq 0 \\
	0 & x<0
\end{cases}
\]
Βρίσκουμε την αθροιστική συνάρτηση κατανομής: 
\[ F(x) = \int_0^x \frac{x'}{\sigma^2} e^{-x'^2/2\sigma^2}dx' = \left[-e^{-x'^2/2\sigma^2}\right]_0^x = 1 - e^{-x^2/2\sigma^2} \]
Για να βρούμε τον μετασχηματισμό: 
\[ F(x) = P(X\leq x) = P(T(U) \leq x) = P(U \leq T^{-1}(x)) = T^{-1}(x)\]
Συνεπώς, πρέπει: 
\[ T(x) = F^{-1}(x) = \left[\ln{\left(\frac{1}{(1-x)^{2\sigma^2}} \right)}\right]^{1/2} \]
Φτιάχνουμε τυχαίους αριθμούς που ακολουθούν την ομοιόμορφη κατανομή στο [0,1], ορίζουμε τις κατάλληλες συναρτήσεις και εφαρμόζουμε τον μετασχηματισμό(ο κώδικας θα επισυναφθεί σε αρχείο .ipynb). \\
Στο τέλος, για να υπολογίσουμε το ποσοστό των κυμάτων που έχουν ύψος $3m<x<5m$:
\[ P(3<x<5) = \int_3^5f(x)dx = F(5) - F(3) \approx 0.28 \]
\end{document}
